\chapter{GIT PHÂN NHÁNH}
\section{Giới thiệu về phân nhánh trong Git}

Hầu như mọi VCS đều có một số hình thức hỗ trợ phân nhánh. Phân nhánh có nghĩa là bạn tách khỏi dòng phát triển chính và tiếp tục làm việc mà không làm hỏng dòng chính đó. Trong nhiều công cụ VCS, đây là một quá trình khá tốn kém, thường yêu cầu phải tạo một bản sao mới của thư mục mã nguồn, có thể mất nhiều thời gian đối với các dự án lớn. Phân nhánh được coi là tính năng cốt lõi và quan trong nhất của Git, nó khiến Git trở nên khác biệt so với các VCS khác. Tại sao nó lại đặc biệt như vậy? Cách phân nhánh của Git cực kỳ nhẹ, khiến các hoạt động phân nhánh gần như diễn ra ngay lập tức và việc chuyển đổi qua lại giữa các nhánh nói chung cũng nhanh như vậy. Không giống như nhiều VCS khác, Git khuyến khích các quy trình làm việc phân nhánh và hợp nhất thường xuyên, thậm chí nhiều lần trong một ngày. 

Việc thực hiện phân nhánh trong Git cho phép ai đó phân nhánh khỏi nhánh chính (nơi mã nguồn của production không bị ảnh hưởng) và làm việc trên một tính năng hoặc sửa lỗi độc lập với phần còn lại của dự án. Sau khi hoàn thành công việc, có thể hợp nhất nhánh trở lại nhánh chính chỉ bằng một cú nhấp chuột. Các nhánh cung cấp một môi trường biệt lập để nhà phát triển làm việc. Việc này cũng hữu ích để tổ chức các cấp độ chi tiết khác nhau. Ví dụ, trong Git-Flow, có ba nhánh quan trọng: master, staging và develop. Master là nhánh cha, đại diện cho trạng thái của dự án được phát hành công khai (đã triển khai). Nhánh staging là nơi thực hiện QA trước khi phát hành. Develop là nhánh nơi các nhà phát triển làm việc trên dự án. Tất cả các nhánh tính năng đều được phân nhánh từ nhánh develop.

Để có thê hiểu về thêm về cách mà Git phân nhánh, cần phải quay lại "Chương 1: Giới thiệu về Git" để có thể xem lại cách mà Git lưu trữ dữ liệu. Như đã biết, Git không hề lưu trữ thông tin dưới dạng một tập hợp các tệp và các thay đổi được thực hiện cho mỗi tệp theo thời gian, mà thay vào đó lưu trữ một loạt nhưng "ảnh chụp nhanh". Khi bạn thực hiện một commit, Git lưu trữ một đối tượng commit có chứa một con trỏ đến "ảnh chụp nhanh" của nội dung bạn đã thực hiện. Đối tượng này cũng chứa tên và địa chỉ email của tác giả, tin nhắn mô tả và con trỏ đến commit hoặc những commit được thực hiện ngay trước commit này (cha của commit): sẽ không có commit phía trước của commit đầu tiên, sẽ có một commit ngay phía trước một commit thông thường và có nhiều commit ngay phía trước của một commit được thực hiện thông qua việc gộp hai hay nhiều nhánh lại với nhau.


